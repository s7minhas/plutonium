\documentclass[12pt,pdflatex,authoryear]{elsarticle}

%%%%%%%%%%%%%%%%%%%%%%%%%%%%%%%%%%%%%%%%%%%%%%%%%%
%%%%%%%%%%%%%%%%%%%% PREAMBLE %%%%%%%%%%%%%%%%%%%%
%%%%%%%%%%%%%%%%%%%%%%%%%%%%%%%%%%%%%%%%%%%%%%%%%%

% -------------------- defaults -------------------- %
% load lots o' packages

% references
\usepackage{natbib}

% Fonts
\usepackage[default,oldstyle,scale=0.95]{opensans}
\usepackage[T1]{fontenc}
\usepackage{ae}
% to colorize links in document. See color specification below
\usepackage[pdftex,hyperref,x11names]{xcolor}
% load the hyper-references package and set document info
\usepackage[pdftex]{hyperref}

% Generate some fake text
\usepackage{blindtext}

% layout control
\usepackage{geometry}
\geometry{verbose,tmargin=1.25in,bmargin=1.25in,lmargin=1.1in,rmargin=1.1in}
\usepackage{parallel}
\usepackage{parcolumns}
\usepackage{fancyhdr}

% math typesetting
\usepackage{array}
\usepackage{amsmath}
\usepackage{amssymb}
\usepackage{amsfonts}
\usepackage{relsize}
\usepackage{mathtools}
\usepackage{bm}
\usepackage[%
decimalsymbol=.,
digitsep=fullstop
]{siunitx}

% restricts float objects to be inserted before end of section
% creates float barriers
\usepackage[section]{placeins}

% tables
\usepackage{tabularx}
\usepackage{booktabs}
\usepackage{multicol}
\usepackage{multirow}
\usepackage{longtable}

% to adapt caption style
\usepackage[font={small},labelfont=bf]{caption}

% footnotes at bottom
\usepackage[bottom]{footmisc}
  % \renewcommand{\footnotelayout}{\doublespacing} % set spacing in footnotes
   \newlength{\myfootnotesep}
   \setlength{\myfootnotesep}{\baselineskip}
   \addtolength{\myfootnotesep}{-\footnotesep}
   \setlength{\footnotesep}{\myfootnotesep} % set spacing between footnotes

% to change enumeration symbols begin{enumerate}[(a)]
\usepackage{enumerate}

% to make enumerations and itemizations within paragraphs or
% lines. f.i. begin{inparaenum} for (a) is (b) and (c)
\usepackage{paralist}

% graphics stuff
\usepackage{subfig}
\usepackage{graphicx}
\usepackage[space]{grffile} % allows us to specify directories that have spaces
\usepackage{placeins} % prevents floats from moving past a \FloatBarrier
\usepackage{tikz}

% sideway figures
\usepackage{rotating}
\usepackage{lscape}

% Spacing
\usepackage[doublespacing]{setspace}

% -------------------------------------------------- %


% -------------------- page template -------------------- %

\setlength{\headheight}{15pt}
\setlength{\headsep}{20pt}
\pagestyle{fancyplain}

\fancyhf{}

\lhead{\fancyplain{}{}}
\chead{\fancyplain{}{Civilian Victimization}}
\rhead{\fancyplain{}{\today}}
\rfoot{\fancyplain{}{\thepage}}

% ----------------------------------------------- %

% -------------------- customizations -------------------- %

% easy commands for number propers
\newcommand{\first}{$1^{\text{st}}$}
\newcommand{\second}{$2^{\text{nd}}$}
\newcommand{\third}{$3^{\text{rd}}$}
\newcommand{\nth}[1]{${#1}^{\text{th}}$}

% easy command for boldface math symbols
\newcommand{\mbs}[1]{\boldsymbol{#1}}

% command for R package font
\newcommand{\pkg}[1]{{\fontseries{b}\selectfont #1}}

% approx iid
\newcommand\simiid{\stackrel{\mathclap{\normalfont\mbox{\tiny{iid}}}}{\sim}}

% -------------------------------------------------------- %

%%%%%%%%%%%%%%%%%%%%%%%%%%%%%%%%%%%%%%%%%%%%%%%%%%
%%%%%%%%%%%%%%%%%%%% DOCUMENT %%%%%%%%%%%%%%%%%%%%
%%%%%%%%%%%%%%%%%%%%%%%%%%%%%%%%%%%%%%%%%%%%%%%%%%

% remove silly elsevier preprint note
\makeatletter
\def\ps@pprintTitle{%
 \let\@oddhead\@empty
 \let\@evenhead\@empty
 \def\@oddfoot{}%
 \let\@evenfoot\@oddfoot}

\def\input@path{
	{/Volumes/Samsung_X5/Dropbox/Research/victimization/graphics/},
	{C:/Users/Owner/Dropbox/Research/victimization/graphics/},
	{C:/Users/herme/Dropbox/Research/victimization/graphics/},
	{C:/Users/S7M/Dropbox/Research/victimization/graphics/},
  {/Users/s7m/Dropbox/Research/victimization/graphics/},
  {/Users/dorffc/Dropbox/Research/nothingbutnet/victimization/graphics/},{C:/Users/pjb15180/Dropbox/plutonium/graphics/},
  {/Users/maxgallop/Dropbox/victimizationMex/graphics/}
}

\graphicspath{
	{/Volumes/Samsung_X5/Dropbox/Research/victimization/graphics/},
  {C:/Users/Owner/Dropbox/Research/victimization/graphics/},
	{C:/Users/herme/Dropbox/Research/victimization/graphics/},
	{C:/Users/S7M/Dropbox/Research/victimization/graphics/},
  {/Users/s7m/Dropbox/Research/victimization/graphics/},
  {/Users/dorffc/Dropbox/Research/nothingbutnet/victimization/graphics/},
  {/Users/maxgallop/Dropbox/plutonium/graphics/}
}

\makeatother

\begin{document}

% saying hello ----------------------------------------------- %
\thispagestyle{empty}
\begin{frontmatter}

\title{Decisive or Distracted: the Effects of United States Constraint on Security Networks\tnoteref{t1}}


\author[msu]{Ha Eun Choi}
\ead{choiha3@msu.edu}
\author[duke]{Scott de Marchi}
\ead{scott.demarchi@gmail.com}
\author[strath]{Max Gallop}
\ead{max.gallop@strath.ac.uk}
\author[msu]{Shahryar Minhas}
\ead{minhassh@msu.edu}

\address[strath]{University of Strathclyde, Glasgow, 16 Richmond St., Glasgow, UK G1 1XQ}
\address[msu]{Department of Political Science, Michigan State University, East Lansing, MI 48824, USA}
\address[duke]{Department of Political Science, Duke University, Durham, NC 27703, USA}


\begin{abstract}
\singlespacing{

The United States spent much of the last two decades involved in a pair of wars in the Middle East. These actions required large commitments, not just of blood and treasure, but also of attention and resolve. We thus try to understand the consequences of what one could call either US distraction or a demonstration of US commitment. In particular, we look at how third parties respond to the US's focus on conflicts in one region of the world, in terms of their cooperation in the economic and diplomatic spheres. To do so, we develop a novel measure of US distraction, as well as novel network based measures of economic and diplomatic cooperation or alignment. We seek to test the theory that when the US is more distracted, other states will be more likely to cooperate with the United States's principle rival, China. We find that such cooperation is more likely in the diplomatic sphere than the economic one. However, when we look at how different types of states react to US distraction, we find that across both measures democracies generally respond by moving farther away from China, while non-democracies move closer to the PRC. \\

\raggedright keywords: methods; hegemonic distraction; Sino-US relations; state preferences}

% Can we say 'civil war'?

\vspace{7mm}
%\noindent \textbf{Word Count}: 9822
\end{abstract}

\tnotetext[t1]{\noindent Alphabetical order signifies equal authorship, all mistakes are our own. Replication material and instructions will be made available at \url{https://github.com/s7minhas/plutonium}.}

\end{frontmatter}
% ----------------------------------------------- %

\newpage\setcounter{page}{1}

\section*{Introduction}
 \begin{itemize}
\item Basic question: Does US distraction due to wars, particularly in the Middle East lead to adverse outcomes in the rest of the world.  
 \item Contributions:  
 \begin{enumerate}
 \item. Novel and superior measure of economic alignment  
  \item  Novel measure of US distraction  
  \item. Findings about how US distraction leads states to cozy up/distance themselves from China  
  \end{enumerate}
  \item Can use US withdrawal from Afghanistan and implications for US/China competition as an easy hook 
  \end{itemize} 
\subsection*{Literature Review}
\begin{enumerate}
\item Measure of US distraction -- ????? 
 \item Measure of Alignment  
 \begin{itemize}
 \item Signorino S scores
 \item Axelrod and Bennett
 \item Gartzke UN voting S scores
 \item Bailey, Voeten, Strezhnev ideal points
 \item Gallop and Minhas
 \item McManus and Neiman
 \end{itemize}
\end{enumerate}

\section*{Theoretical Intuition}
US distraction, as measured by a larger share of resources and battle deaths in the Middle East, could be argued to have two divergent effects. On the one hand, we could think of US attention and resources as scarce, and the claim would be that while the US is distracted, say with a conflict in Afghanistan, they have less bandwidth and freedom of action to respond to a crisis in East Asia, we call this \emph{the Distraction Model}. An alternative perspective relies on work that has been done on reputation and credibility, and this perspective argues that by spending blood and treasure in Iraq or Afghanistan, the United States demonstrates their resolve and willingness to spend blood and treasure elsewhere, we call this the \emph{Credibility Model}.

\subsection*{Distraction Model}
The basic insight of the Distraction Model is that both resources and attention are finite. The United States is a historically rich and powerful country, but that each successive conflict or intervention will have less resources, less oversight, and less political capital, and so all else being equal, the United States is less likely to get involved in a third (or fourth or tenth) conflict while it is, for example, involved in occupation and counterinsurgency in Iraq and Afghanistan. This has important implications for the behavior of third parties--if you are considering actions contrary to the interests of the United States, whether this be development of illicit weapons systems, invasion of a neighboring state, or simply a country choosing to align more closely with the US's strategic rivals, you are less likely to face consequences the more the United States is distracted. When applied to our measurement of state affinity, this naturally leads to our first hypothesis.

\emph{Hypothesis 1: When the United States is more distracted, states will align more closely with the People's Republic of China.}

\subsection*{Credibility Model}
The second model draws on a long history of research about credibility and resolve. The basic idea behind this model is that we do not know a state's willingness to take costly actions (for example kinetic military actions or coercive economic measures) to achieve a given goal, because the ration of the cost of these actions to their benefit is unobservable. However, we can make inferences about states' willingness to pursue costly actions using their past behavior--so if a state is willing to spend blood and treasure in one context, we should increase our belief that they are willing to do so in another context, and our measure of US distraction is also a measure of US demonstration of resolve. Now an important caveat here is that credibility and resolve are contextual: states are more willing to pay a high cost to avoid conquest by a genocidal tyrant than they would be to secure lower tariffs on agricultural goods. That being said, this leads us to a diametrically opposite view of how third parties will respond to our measure of US distraction. If the US demonstrates their willingness to bear heavy costs to achieve more congenial political outcomes in one context, we should update our belief about their willingness to bear costs in other contexts, leading to our second hypothesis.

\emph{Hypothesis 2: When the US is more distracted, they will have demonstrated more resolve, and states will be less willing to align closely with the People's Republic of China.}

\textcolor{red}{Should we discuss our theory of heterogeneous effects here with its own hypothesis, or wait for the results section?}

\section*{Methods}

As noted above, we have developed a measure of US constraint to capture a variety of mechanisms that might limit US action. In broad terms, we believe (based on SME input and prior research) that constraint is a function of three possible sources:

\begin{enumerate}
    \item Active conflicts that the US is involved in with an emphasis on those conflicts that represent significant materiel commitment and US casualties
    \item US force commitments around the globe
    \item US political or economic distraction caused by either foreign or domestic shocks.
\end{enumerate}

To population these categories, we relied on sixteen variables distributed across the following broader categories:

\begin{table}[h]
\centering
\begin{tabular}{ll}
Variables & Source \\
\hline
Defense spending  & World Bank, US DoD \\
Troop levels, by region   & US DoD, Kane \\
US Casualties   & US DoD, Kane  \\
Market Crises   &  Frieden and Lake\\
Economic variables   &  Trading Economics \\
\end{tabular}
\caption{Variables for generation of US Distraction Index}
\end{table}

The main source of constraint that is incomplete in the above framework is domestic political crises (i.e., apart from a response to economic shocks, which is captured).\footnote{Prior work has shown that political crises of this sort are rare / not likely consequential \citep{frieden:2017} and we leave the inclusion of political measures to future work.}

A simple latent variable model of the above variables produced three features, representing active US conflicts (F1), US defense spending / commitments (F2), and the economic shocks (F3). A PCA with three latent variables explained 89\% of the variance in the raw data, which is very good. Scree plots / an examination of eigenvalues supported the use of three latent variables:

\begin{figure}[h]
    \centering
    \includegraphics{scree.png}
    \caption{Scree plot of constraint PCA}
    \label{fig:my_label}
\end{figure}

A possible fourth factor, which we did not include, focuses on variance from GDP growth and changes in the stock market.

Graphs of the three factors through time show considerable variance that tracks with our priors on United States military conflicts and economic shocks over the past two decades:

\begin{figure}[ht]
    \centering
    \includegraphics[width=1\textwidth]{facViz.pdf}
    \caption{PCA Viz.}
    \label{fig:facViz}
\end{figure}
\FloatBarrier

\subsection*{Measuring Alignment}

To test our theories, we require a measure of how closely states are aligned and how these relationships change over time. One way to do this would simply be to look at a raw measure of cooperation between states, but this has a number of issues. First, states have very different overall levels of cooperative activity – e.g., the fact that there is much more trade between the United States and China than between China and Mongolia does not mean that there is closer alignment between those two states. Rather, it means that the US and China trade more with everybody. Second, there would be difficulties of left censorship. If we take trade (for example) to be the consequence of a latent measure of economic alignment, it is important to distinguish between states that do not trade with each other because of antipathy versus neutrality. E.g., the US has little with Cuba or Iran due to antipathy; it has almost no trade with Slovenia for quite different reasons. We believe that both of these issues can be ameliorated by treating measures of cooperation as a form of relational data using a network approach to infer our latent measure of alignment. This is named the Latent Factor Model (LFM).

We use two different raw measures of cooperation as input data: the balance of trade between states (as measured by the IMF) and similarity in states’ UN voting records. For the balance of trade, we take the volume trade between states for a given dyad and divide that by the total volume for one state in the dyad. These form the links in our network of economic alignment. For our measure of diplomatic alignment, we look at the percent of votes at the United Nations General Assembly in which states voted in the same way. We use these two networks to estimate state affinity using the Latent Factor Model.\footnote{See (cite) for other attempts to use the latent factor model to infer alignments in different contexts.}  In particular, we take X years of balance of trade data / UN voting data and each of them forms one slice of our multilayer network.

The latent factor model is a network model that is designed to account for three different orders of interdependencies in relational data. First, it accounts for the tendency of some actors to trade more and agree to more economic agreements by including sender and receiver random effects. Second it accounts for the fact that economic cooperation is often reciprocal in the composition of the error term. Finally, the area that sets the LFM apart from other network estimators is how it handles third order dependencies. Two particular types of third order dependencies which the LFM can handle are homophily – the tendency for actors that share an unobserved characteristic – to interact more with each other, and stochastic equivalence, the idea that actors which play similar roles in a network are more likely to cooperate with the same third party. The LFM handles these third order dependencies with a multiplicative random effect based on the Singular Value Decomposition.\footnote{This effect needs to be multiplicative because by multiplying random variables, we can preserve the third order residuals which would have 0 expectation if they were simply added.} This third order term is useful in allowing us to cope with left censoring in this data, since we can use their trade with common third parties to determine if they are in the same realm of the global economic network, or if their opposition runs deeper. The equations underlying the latent factor model are as follows:

\begin{align}
	\begin{aligned}
		y_{ij} \;=\; f(\theta_{ij}) &\text{, where } \\
		\theta_{ij} \;=\;& \bm\beta_{d}^{\top} \mathbf{X}_{ij} + \bm\beta_{s}^{\top} \mathbf{X}_{i} + \bm\beta_{r}^{\top} \mathbf{X}_{j}  \\
		& + a_{i} + b_{j} + \epsilon_{ij} \\
		& + \mathbf{u}_{i}^{\top} \mathbf{D} \mathbf{v}_{j}   \\
	\label{eqn:ame}
	\end{aligned}
\end{align}


\begin{align}
	\begin{aligned}
		\{ (a_{1}, b_{1}), \ldots, (a_{n}, b_{n}) \} &\simiid N(0,\Sigma_{ab}) \\
		\{ (\epsilon_{ij}, \epsilon_{ji}) : \; i \neq j\} &\simiid N(0,\Sigma_{\epsilon}), \text{ where } \\
		\Sigma_{ab} = \begin{pmatrix} \sigma_{a}^{2} & \sigma_{ab} \\ \sigma_{ab} & \sigma_{b}^2   \end{pmatrix} \;\;\;\;\; &\Sigma_{\epsilon} = \sigma_{\epsilon}^{2} \begin{pmatrix} 1 & \rho \\ \rho & 1  \end{pmatrix}
	\label{eqn:srm}
	\end{aligned}
\end{align}

In particular, we argue that the $\mathbf{u_i}^{T}\mathbf{D}\mathbf{v_j}$ term, which is included in the model to capture third order dependencies, also is useful for us as a measure of economic alignment. We run an LFM without covariates on the economic data, and then take this term for every pair of countries in every year, as a measure of economic alignment and one with UN voting similarity as a measure of diplomatic alignment.\footnote{We ran models with both a 2 and a 5 dimensional latent factor space, and found the results to be relatively consistent, and so for the sake of clarity, we focus on the easier to interpret 2 dimensional results.}

\subsection*{Face Validity}

With a measure like this, it is important to investigate whether it is giving us leverage over the unobserved relationships that we are trying to estimate. We examine face validity in two ways – first by looking at the overall network of relationships uncovered, and then by looking in more detail at the time series of certain prominent relationships.

The latent factor model which underpins our measures of relationships maps each state into a k dimensional latent vector space. States that have their vectors pointed in similar directions are more likely to influence each other and common third parties, whereas states whose vectors point in opposite directions have limited influence on each other, and in many cases antipathy. We plot the overall distribution of the network in both 2000 and 2019 (for UN voting data) or 2020 (for trade data), in figures 5-8.

\begin{figure}[ht]
	\begin{tabular}{cc}
	\includegraphics[width=.45\textwidth]{un00.pdf} &
	\includegraphics[width=.45\textwidth]{un19.pdf} \\
	\end{tabular}
	\caption{Visualization of multiplicative effects for our measure of diplomatic influence in 2000 (left) and 2019 (right). Each circle designates a country and the color corresponds to the legend at the center of the visualization. Countries that cluster together are those that were found by the model to have similar sending patterns, meaning that they tend to influence one another.}
	\label{fig:unUV}
\end{figure}
\FloatBarrier

\begin{figure}[ht]
	\begin{tabular}{cc}
	\includegraphics[width=.45\textwidth]{trade00.pdf} &
	\includegraphics[width=.45\textwidth]{trade20.pdf} \\
	\end{tabular}
	\caption{Visualization of multiplicative effects for our measure of trade influence in 2000 (left) and 2020 (right). Each circle designates a country and the color corresponds to the legend at the center of the visualization. Countries that cluster together are those that were found by the model to have similar sending patterns, meaning that they tend to influence one another.}
	\label{fig:tradeUV}
\end{figure}
\FloatBarrier


There are a few immediate takeaways from these networks – first is that the diplomatic influence measure based on UN voting shows three pretty clear clusters: the US and Israel (and Canada in 2020) are relatively isolated, but generally close to a larger cluster with most of the other major European powers, along with Japan, Australia, and New Zealand. The third cluster contains the vast majority of the global South. This is true whether we are looking at a latent factor model with 2 or 5 dimensional latent factors. The network for trade in 2000 paints a somewhat different story. While we still see clustering of many of the major western powers, there is a much stronger role played by geography here – the US is close to many other states in the Americas, and Russia is close to many European states. We also, as one might expect, see much closer alignment between the US and China. In 2020, the geographic clustering remains, but Russia has drifted away from Europe, and US/China economic relations are somewhat less close. These figures show that the measures of diplomatic and economic influence correspond to many of our intuitions about influence in world politics, while also maintaining important and novel characteristics based on the data used to generate them.

We also test the face validity of these measures by looking at how they characterize a trio of important relationships. We choose the relationship between the US and UK, which we expect to be generally close and amicable, and the US’s relations with its two major competitors China and Russia. As shown in figure 9, this measure captures the general tenor of the relationships – the US and UK have a consistently positive relationship, whereas the relationship the US has with both China and Russia, based on UN voting, is characterized as more adversarial – the time series interestingly points to generally positive relationships in the immediate aftermath of 9/11, which deteriorate precipitously starting in 2003 with the Iraq war, and while there are some marginal improvements, the relationship stays quite negative. On the other hand, as shown in figure 10, while our measure of economic influence pinpoints the positive US/UK relationship, and the negative US/Russia relationship, it finds that the US has a relationship with China that is at times even more closely aligned than that with the United Kingdom. This makes a degree of sense given that the volume of US/China trade dwarfs the trade in the so-called special relationship.

\begin{figure}[ht]
\centering
\includegraphics[width=1\textwidth]{distViz.pdf}
\caption{ Cosine distance in latent factor space.
}
\label{fig:distViz}
\end{figure}
\FloatBarrier



\section*{Downstream Modeling}
Our unit of observation here is at the country year level. We do this for 118 countries, for every year between 2000 and 2020. The countries excluded from the analysis are 1) those countries where the IMF has no data on their trade flows, 2) the United States and China, since we are interested in the effects of US distraction on economic alignment with China. To generate our dependent variable, we take our measure of a country's economic alignment with China, and we measure the change in it from one year to the next, to see how countries move towards or away from China given the distraction or demonstrated resolve of the United States.

Our key independent variable, as discussed in section \ref{}, is our measures of US distraction based either on active US conflicts (called f1) and US defense expenditures and force commitments (called f2). For each country-year observation, we include the value of US distraction in the previous year to see how it relates to that country's alignment with China.

We also include a number of country level controls that might influence a country's closeness of economic alignment with China. Following the insights of the gravity model of trade, we control for a country's population, their GDP, and the distance between their capital and Beijing. To account for political factors that might make a country more or less closely aligned with China, we include both Polity's measure of a country's level of democracy (the intuition being that autocratic regimes will be more close to other autocracies, like China, and democracies will be less close, all else being equal), and a measure of distance between a country's ideal point, and China's using \citet{bailey:voeten:strezhnev:year}'s measures of state preference drawn from UN voting data. We also include random effects using the UN's 23 region's to account for potentially systemic differences in tendencies of countries to align with the US and China.

\subsection*{Main Results}
\subsection*{Heterogeneous Effect}

\section*{Conclusion}

In this paper, we develop a theory of how constraints faced by the US have realigned cooperation networks in the international system. Specifically, during times of US distraction, we find that the level of cooperation between China and non-democratic countries has notably increased. To arrive at this finding we not only developed a factor based measures of US distraction, but also a network based approach to understanding how cooperation with China is changing in the diplomatic and economic realms. Our distraction measure show notable variation over time and we provide face validity evidence for our estimated measures of cooperation. The implications from our findings highlight what simply examining conflict data cannot. China's role in the international	system is changing in measurable ways and its ability to potentially exert influence has become more notable in periods of US distraction. 


% Bib stuff
\clearpage
\singlespacing

\bibliography{master}
% \bibliographystyle{elsarticle-harv}\biboptions{authoryear}
\bibliographystyle{apsr}

\appendix

\end{document}
