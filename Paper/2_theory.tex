\section*{Theoretical Intuition}
Constraints on the United States, represented by military entanglements and economic shocks, could have two divergent effects. On the one hand, one could think of US attention and resources as scarce, and the mechanism would be that while the US is distracted (e.g., with a conflict in Afghanistan or Iraq), they have less bandwidth and freedom of action to respond to a crisis in East Asia.  Our expectation is that both types of constraint – military and economic – would have the same effect.  We call this \emph{the Distraction Model}. 

An alternative mechanism relies on research that has been done on reputation and credibility and this perspective argues that by spending blood and treasure in contexts like Iraq or Afghanistan, the United States demonstrates their resolve and willingness to commit elsewhere.  Obviously, this causal mechanism relies on military constraint only and does not depend on economic conditions.  We call this \emph{the Credibility Model}.  

\subsection*{The Distraction Model}
The basic insight of the Distraction Model is that both resources and attention are finite \citep{deutsch:singer:1964, haass:2008, bilmes:2013}. The United States is a historically rich and powerful country, but each successive conflict or intervention will reduce available resources, especially in an increasingly multipolar world. All else equal, the United States may have been less likely to get involved in a second (or third or …) conflict while it was, for example, involved in Iraq and Afghanistan. This has important implications for the behavior of third parties, especially if they are considering actions contrary to the interests of the United States. For example, a state developing illicit weapons systems, invading   a neighboring state, or simply aligning more closely with the US’s strategic rivals, may face fewer consequences while the United States is distracted. 

When applied to our measurement of state affinity capturing both military and economic conditions, this naturally leads to our first hypothesis. In the research presented here, our focus here is on third party actors and their relationship with China (i.e., a US adversary). We are obviously not including the entire range of hostile actions that third parties might take:

\emph{Hypothesis 1: When the United States is more distracted, states will align more closely with the People’s Republic of China}

\subsection*{Credibility Model}

The second model draws on a long history of research about credibility and resolve \citep{schelling:1966,walter:2006,crescenzi:2007,gibler:2008, weisiger:yarhi-milo:2015}. The idea motivating this model is that we do not know a state’s willingness to take costly actions (for example, kinetic military actions or coercive economic measures) to achieve a goal because the ratio of the cost of these actions to their benefit is unobservable. We can, however, make inferences about states’ willingness to pursue costly actions using their past behavior as a guide.  Put simply, If the United States demonstrates willingness to bear heavy costs to achieve desirable political outcomes in one context, third party actors should update beliefs about the United States’ willingness to bear costs in other contexts.  Accordingly, our measure of US military entanglement is also a measure of US demonstration of resolve. 

An important caveat is that credibility and resolve are contextual: states are more willing to pay a high cost to avoid conquest by a genocidal tyrant than they would be to secure lower tariffs on agricultural goods. That said, this leads us to a diametrically opposite view of how third parties will respond to our measure of US constraint. This leads to our second hypothesis – as above, we are focused on third party relations with China:

\emph{Hypothesis 2: When the US is constrained due to military commitments, they will have demonstrated more resolve, and states will be less willing to align closely with the People’s Republic of China.}



