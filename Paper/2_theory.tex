\section*{Theoretical Intuition}
US distraction, as measured by a larger share of resources and battle deaths in the Middle East, could be argued to have two divergent effects. On the one hand, we could think of US attention and resources as scarce, and the claim would be that while the US is distracted, say with a conflict in Afghanistan, they have less bandwidth and freedom of action to respond to a crisis in East Asia, we call this \emph{the Distraction Model}. An alternative perspective relies on work that has been done on reputation and credibility, and this perspective argues that by spending blood and treasure in Iraq or Afghanistan, the United States demonstrates their resolve and willingness to spend blood and treasure elsewhere, we call this the \emph{Credibility Model}.

\subsection*{Distraction Model}
The basic insight of the Distraction Model is that both resources and attention are finite. The United States is a historically rich and powerful country, but that each successive conflict or intervention will have less resources, less oversight, and less political capital, and so all else being equal, the United States is less likely to get involved in a third (or fourth or tenth) conflict while it is, for example, involved in occupation and counterinsurgency in Iraq and Afghanistan. This has important implications for the behavior of third parties--if you are considering actions contrary to the interests of the United States, whether this be development of illicit weapons systems, invasion of a neighboring state, or simply a country choosing to align more closely with the US's strategic rivals, you are less likely to face consequences the more the United States is distracted. When applied to our measurement of state affinity, this naturally leads to our first hypothesis.

\emph{Hypothesis 1: When the United States is more distracted, states will align more closely with the People's Republic of China.}

\subsection*{Credibility Model}
The second model draws on a long history of research about credibility and resolve. The basic idea behind this model is that we do not know a state's willingness to take costly actions (for example kinetic military actions or coercive economic measures) to achieve a given goal, because the ration of the cost of these actions to their benefit is unobservable. However, we can make inferences about states' willingness to pursue costly actions using their past behavior--so if a state is willing to spend blood and treasure in one context, we should increase our belief that they are willing to do so in another context, and our measure of US distraction is also a measure of US demonstration of resolve. Now an important caveat here is that credibility and resolve are contextual: states are more willing to pay a high cost to avoid conquest by a genocidal tyrant than they would be to secure lower tariffs on agricultural goods. That being said, this leads us to a diametrically opposite view of how third parties will respond to our measure of US distraction. If the US demonstrates their willingness to bear heavy costs to achieve more congenial political outcomes in one context, we should update our belief about their willingness to bear costs in other contexts, leading to our second hypothesis.

\emph{Hypothesis 2: When the US is more distracted, they will have demonstrated more resolve, and states will be less willing to align closely with the People's Republic of China.}

\textcolor{red}{Should we discuss our theory of heterogeneous effects here with its own hypothesis, or wait for the results section?}