\section*{Introduction}

An abiding concern of the interstate security literature has been conflict between nations.  To a large degree, this approach has been facilitated by the availability of data – notably, the Correlates of War dataset (focused primarily on militarized interstate disputes \citealp{palmer:etal:2021}) and the Alliance Treaty Obligations and Provisions project (focused on formal alliances between nations \citealp{leeds:etal:2002}).  Without risk of exaggeration, explaining the causes of military conflict – or predicting future conflict – is the biggest show in town \citep{ward:etal:2013}.  Substantive findings from this literature continue to be well-cited, ranging from democratic peace  \citep{maoz:russett:1993} to \citet{bennett:stam:2003}'s comprehensive approach to explaining the causes of war.

While contributions made by this vast literature are important, questions about generalizability remain \citep{jenke:gelpi:2017}.  What researchers would like to do is to limit studies of war to time periods in which the causes of war spring from the same data generating process.  For example, the ability of nations to project force and maintain units in the field has changed dramatically over time, and as a result, any dyadic examination of war should limit models to observations of conflict that are independent and identically distributed (IID).  Unfortunately (for research, though not for world peace), only ~4\% of dyad years have a military conflict and the majority of those are continuations of conflict rather than the initiation of conflict (which are two different data generating processes – see \citet{demarchi:etal:2004}).  Within any given regime, many of these conflicts are geographically concentrated – in the post-WW2 period, disproportionately many of the conflicts occur in the Middle East.

We are thus faced with two competing constraints: either we broaden the sample and inappropriately treat non-IID observations together, or we can limit our sample to a particular regime but end up with geographically limited and rare event data.  There are, in short, no easy answers to this problem.

War is an event of obvious importance, but it is difficult to study given its infrequency and the changing nature of warfare across time.  In this article, we propose to study the other side of the coin: cooperation between states.  While the causal mechanisms that lead to war and cooperation are likely distinct, it is reasonable to assume that, all else equal, states that are embedded in a cooperative relationship are much less likely to engage in military conflict.  By understanding the causes of cooperation between states we thus sidestep the statistical issues involved with studying conflict directly and can instead focus on the more plentiful data generated by varying levels of cooperation between states.

Importantly, the relatively plentiful supply of data on cooperation between states also gives us considerable flexibility in the types of questions we can answer.  There is genuine year to year variance in the relationships between states and this allows us to examine a broader set of mechanisms that explain cooperation.

Here, we focus on a particular type of shock to the international system: the level of economic and military constraint faced by the United States.  The United States, especially in the last two decades, has been a pivotal actor in the international system; constraints on the United States’ capacity should impact the actions taken by other states – both with respect to cooperation with the United States directly but also cooperation with other actors of interest such as China. We are thus interested in a relatively straight-forward model that investigates the proximity of cooperative relationships between state as a function of United States economic and military constraint.

Just as alliances \citep{warren:2010, cranmer:etal:2015a} and conflict \citep{maoz:2012a, ward:etal:2007} occur in networks between states, so does cooperation.  Accordingly, we rely on a latent factor model (LFM) to study these interdependencies between states \citep{minhas:etal:2019, hoff:2021}.  Here we employ this general framework to develop a latent measurement of how a state relates to other states in a network context.  The factor analysis we employ seeks to take as an input the interactions that actors have with others across a variety of dimensions and project this onto a low-dimensional space. In many ways this goal is no different than how others have sought to find simpler representations of legislators and bills \citep{poole:rosenthal:1985} or topic models for text \citep{roberts:etal:2016}.

We seek to model state interactions on a network, where there are specific types of patterns that often occur and that should be captured when reducing the dimensionality of a network. One such pattern is stochastic equivalence.  Stochastic equivalence refers to the idea that there are communities of nodes in a network, and actors within a community act similarly towards those in other communities. Thus, the community membership of an actor provides us with information on how that actor will act towards others in the network. Put more concretely, a pair of actors $ij$ are stochastically equivalent if the probability of i relating to, and being related to, by every other actor is the same as the probability for $j$ \citep{anderson:etal:1992}. For example, in an international cooperation network, we might see relatively isolated rogue states, like North Korea and Iran, as being stochastically equivalent, because they have limited cooperation with states like China and Russia, and generally conflictual relations with rich Western states. Similarly, close allies of the United States (for example Canada and the United Kingdom) will exhibit high degrees of stochastic equivalence, likely to cooperate with other rich Western states, and direct conflictual acts towards said rogue states. This concept simply speaks to the assertion that we can learn something about how an actor will interact with an entire network based on, for example, the existing set of relationships that they are enmeshed in.

An additional dependence pattern that often manifests in networks is homophily -- the tendency of actors to form transitive links. The presence of homophily in a network implies that actors may cluster together because they share some latent attribute. In the context of clustering in alliance relationships, we are likely to find that states like the United States, United Kingdom, and Germany may cluster together because they share some latent state level attribute. We would ignore salient information if we did not use, for example, the United Kingdom's behavior towards third parties, when trying to understand the United States' preference similarity with those parties. Doing so is likely to paint an incomplete picture of the preferences that states share with one another.

The LFM accounts for these higher order dependence patterns and ensures that similarity in preferences is likely to be transitive, for example, if the United States has similar preferences to the United Kingdom, and the United Kingdom to France, the United States' preferences should be relatively close to France's. Further the most useful feature of the LFM for our purpose is that it places actors it summarizes the interdependencies between actors in a relational k-dimensional latent vector space.

Our plan for the rest of the article is straight-forward.  First, we will sketch the two competing theories for how the United States' constraint could affect relationships in the international system.  Second, we will establish our measure of the level of constraint of the United States during the period 2000 – 2020.  Third, we will detail the LFM and output measures for this time period.  Last, we will present the results of our downstream regressions that detail the impact on cooperation as a function of constraint.
