
\section*{Downstream Modeling}
Our unit of observation here is at the country year level. We do this for 118 countries, for every year between 2000 and 2020. The countries excluded from the analysis are 1) those countries where the IMF has no data on their trade flows, 2) the United States and China, since we are interested in the effects of US distraction on economic alignment with China. To generate our dependent variable, we take our measure of a country's economic alignment with China, and we measure the change in it from one year to the next, to see how countries move towards or away from China given the distraction or demonstrated resolve of the United States.

Our key independent variable, as discussed in section \ref{}, is our measures of US distraction based either on active US conflicts (called f1) and US defense expenditures and force commitments (called f2). For each country-year observation, we include the value of US distraction in the previous year to see how it relates to that country's alignment with China.

We also include a number of country level controls that might influence a country's closeness of economic alignment with China. Following the insights of the gravity model of trade, we control for a country's population, their GDP, and the distance between their capital and Beijing. To account for political factors that might make a country more or less closely aligned with China, we include both Polity's measure of a country's level of democracy (the intuition being that autocratic regimes will be more close to other autocracies, like China, and democracies will be less close, all else being equal)\citet{marshall:jaggers:2002}. 

To account for other structural factors, we estimate our models within a hierarchical framework which allows for both the fixed effects discussed above, as well as random effects. In the first set of results, we rely on random effects for country, and later we look at random effects based on a states domestic political institutions.

\subsection*{Main Results}
In the first set of models depicted in \ref{models:1} we see a marked divergence between our measures of diplomatic and economic alignment. As F1 increases, denoting a higher level of US attention and resources devoted to the Middle East, states have a consistent increase in their diplomatic alignment with China. The opposite, however, seems to be the case for economic alignment, as US distraction seems to be associated with an aggregate move away from China.  When we move from one measure of US distraction to another (F2 based on troop deployments and military spending, as seen in \ref{models:2}), we see a similar pattern -- diplomatic alignment with China increases with our measures of US distraction, economic alignment does not. 

For control variables we see a similar division between the two types of dependent variable. Democracies are less likely to be aligned diplomatically with China, but more likely to be aligned economically, and the same is true for rich states. The distance to Beijing has a consistent positive effect across all four models, indicating that there is interestingly, closer alignment between China and more distant states.
\textcolor{red}{SM: can you add some way of depicting these model results, don't care if coeff plot, table, whatever}

\subsection*{Heterogeneous Effect}
One possible explanation for these ambiguous results is that these models show the aggregate effect, but beneath the surface, different states react to US distraction in different ways. To get at this, we modify our hierarchical model to not just allow random intercepts, but also random slopes, and rather than using countries as our random effect, we group states using their polity scores. We do this in two ways: first we use the canonical division into consolidated democracies (Polity Score >= 7), consolidated autocracies (< -6) and mixed regimes (all other states). We also divide country years into 4 quartiles from the least to the most democratic, and in these models we allow a random slope for F1 or F2 in each tranche of the polity score.

For F1 and the canonical polity categories, we find a similar effect across our dependent variables. There is a consistent positive effect of US distraction among autocracies, meaning that autocracies are more aligned with China when the US is distracted, but as we move from consolidated autocracy to mixed regime, and to democracy, this effect shrinks, and in many cases changes direction, such that consolidated democracies, across 3 of the 4 dependent variables become less aligned with China when the US is more distracted.  For F2, there is a consistent positive effect of US distraction for both autocracies and mixed regimes, but the effect again reverses itself in consolidated democracies.\footnote{The result also generally holds when we let the data determine the size of the institutional groupings, rather than prespecifying these canonical groups}

\textcolor{red}{SM: can you add some way of depicting these model results, don't care if coeff plot, table, whatever}