\section*{Methods}
\subsection*{Measuring Distraction}
We have developed a measure of US constraint / distraction to capture a variety of mechanisms that might limit US action.  In broad terms, we believe (based on SME input and prior research) that constraint is a function of three possible sources:
\begin{enumerate}
\item	Active conflicts that the US is involved in with an emphasis on those conflicts that represent significant materiel commitment and US casualties
\item US force commitments around the globe
\item US political or economic distraction caused by either foreign or domestic shocks.
\end{enumerate}
To population these categories, we relied on sixteen variables distributed across the following broader categories:

\begin{itemize}
\item Defense spending	(World Bank, DoD)
\item Troops levels, by region	(DoD, Kane)
\item US Casualties	(DoD, Kane)
\item Market Crises	(Frieden & Lake)
\item Economic variables	(Trading Economics)
\end{itemize}

The main source of constraint that is incomplete in the above is domestic political crises (i.e., apart from a response to economic shocks, which is captured).  Prior work has shown that political crises of this sort are rare / not likely consequential (Frieden, et al., 2017), but we will add an improved measure if we continue the project.  A simple latent variable model of the above variables produced three variables, representing active US conflicts, US defense spending / commitments, and the economy.   

These latent variables define the level of US constraint; in what follows, we will however focus on the first two latent variables because they more narrowly affect the US (and not other members of our sample). \textcolor{red}{Go into more detail about the PCA, and add pictures}.

\begin{figure}
\centering
\includegraphics[height = 8cm, width = 8cm]{distVisFacet}
\end{figure}



\subsection*{Measuring Economic Alignment}
To test our theories, we want a measure of how closely states are aligned economically, and how this changes over time. One way to do this would simply be to look at the raw volume of trade between states, but this has a number of issues. First, states have very different overall levels of trade -- the fact that there is much more trade between the United States and China than between China and Mongolia does not mean that there is closer alignment between those two states, rather it means that they trade more with \emph{everybody}. Second, there would be difficulties of left censorship. If we take trade to be the consequence of a latent measure of economic alignment, it is important to distinguish between states that do not trade with eachother because of antipathy and neutrality: the difference between the US having no trade with Cuba or Iran, and having almost no trade with Slovenia is much greater than the trade data makes it appear. We believe that both of these issues can be ameliorated by treating trade as a form of relational data, and using a network approach, called the Latent Factor Model, to infer our latent measure of economic alignment.

The raw data we use here is the balance of trade between two states (as measured by the IMF), we do this by taking the  volume of states for a given dyad, and dividing that by the total volume for one state in the dyad. These form the links in our network of economic, which we use to estimate state affinity using the Latent Factor Model \citep{hoff:year}. In particular, we take X years of balance of trade data, and each of them forms one slice of our multilayer network. The latent factor model is a network model that is designed to account for three different orders of interdependencies in relational data. First, it accounts for the tendency of some actors to trade more, and agree to more economic agreements by including sender and receiver random effects. Second it accounts for the fact that economic cooperation is often reciprocal in the composition of the error term. Finally, the area that sets the LFM apart from other network estimators is how it handles third order dependencies. Two particular types of third order dependencies which the LFM can handle are homophily -- the tendency for actors that share an unobserved characteristic -- to interact more with each other, and stochastic equivalence, the idea that actors which play similar roles in a network are more likely to cooperate with the same third party. The LFM handles these third order dependencies with a multiplicative random effect based on the Singular Value Decomposition.\footnote{This effect needs to be multiplicative because by multiplying random variables, we can preserve the third order residuals which would have 0 expectation if they were simply added.} This third order term is useful in allowing us to cope with left censoring in this data, since we can use their trade with common third parties to determine if they are in the same realm of the global economic network, or if their opposition runs deeper. The equations underlying the latent factor model are as follows:


\begin{align}
	\begin{aligned}
		y_{ij} \;=\; f(\theta_{ij}) &\text{, where } \\
		\theta_{ij} \;=\;& \bm\beta_{d}^{\top} \mathbf{X}_{ij} + \bm\beta_{s}^{\top} \mathbf{X}_{i} + \bm\beta_{r}^{\top} \mathbf{X}_{j}  \\
		& + a_{i} + b_{j} + \epsilon_{ij} \\
		& + \mathbf{u}_{i}^{\top} \mathbf{D} \mathbf{v}_{j}   \\
	\label{eqn:ame}
	\end{aligned}
\end{align}


\begin{align}
	\begin{aligned}
		\{ (a_{1}, b_{1}), \ldots, (a_{n}, b_{n}) \} &\simiid N(0,\Sigma_{ab}) \\
		\{ (\epsilon_{ij}, \epsilon_{ji}) : \; i \neq j\} &\simiid N(0,\Sigma_{\epsilon}), \text{ where } \\
		\Sigma_{ab} = \begin{pmatrix} \sigma_{a}^{2} & \sigma_{ab} \\ \sigma_{ab} & \sigma_{b}^2   \end{pmatrix} \;\;\;\;\; &\Sigma_{\epsilon} = \sigma_{\epsilon}^{2} \begin{pmatrix} 1 & \rho \\ \rho & 1  \end{pmatrix}
	\label{eqn:srm}
	\end{aligned}
\end{align}

In particular, we argue that the $\mathbf{u}_{i}^{\top} \mathbf{D} \mathbf{v}_{j}$ term, which is included in the model to capture third order dependencies, also is useful for us as a measure of economic alignment. So we run an LFM without covariates on the economic data, and then take this term for every pair of countries in  every year, as a measure of economic alignment. This is calculated to maximize the fit of X years of economic data, which also helps us account for temporal dependencies. In figure \ref{china:map} we show a map with each countries economic alignment with China in 2000 and 2020, in figure \ref{plaus:plots} we show the evolution of our measure for a number of prominent dyads, and in figure \ref{network:deps} we show the network of econic alignment highlighting the role of geographic proximity.



